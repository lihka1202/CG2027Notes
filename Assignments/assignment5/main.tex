\documentclass{article}

\title{Assigment 5}
\author{Muthya Narayanachary Akhil}
\date{February 19th, 2024}

\begin{document}
\maketitle

\section*{Question 1}

\subsection*{Part a}

When Q = 1 and $\overline{Q}$ = 0 and WL = H, this implies that M5 and M1 form a voltage ladder.
Hence the $\overline{Q}$ is deduced by this equation:
\begin{equation}
    \overline{Q} = V_{DD} * \frac{R_{M1}}{R_{M5} + R_{M1}}
\end{equation}
Hence we can observe the following:
\begin{equation}
    1 * \frac{R_{M1}}{R_{M5} + R_{M1}} < 0.4.
\end{equation}

\begin{equation}
    6 * R_{M1} < 4 * R_{M5} = R_{M1} < \frac{2}{3} * R_{M5}
\end{equation}

This implies that $W_{M1}$ should be greater than 1.5 $\mu$m.

\subsection*{Part b}

When Q = 1 and $\overline{Q}$=0, M2 and M3 are off, so the writing part is done by M4 and M6 instead.
M4 and M6 form a voltage ladder, hence voltage at this point can be given to be:
\begin{equation}
    1 * \frac{R_{M6}}{R_{M4} + R_{M6}} < 0.4
\end{equation}

\begin{equation}
    6 * R_{M6} < 4 * R_{M4}
\end{equation}
That is, $W_{M4}$ must be less than $\frac{2}{3}$ $\mu$m.


\section*{Queston 2}
\subsection*{Part a}
As we know,
\begin{equation}
    Q_{overall} = Q_{BL(initial)} + Q_{S(initial)}
\end{equation}
Based on the data we already khow, we can infer the following:
\begin{equation}
    (80fF + 20fF) * V_{final} = 80tF*1
\end{equation}
\begin{equation}
    V_{final} = 0.8V
\end{equation}
Hence the dropping amount is 1 - 0.8 V = 0.2 V

\subsection*{Part b}
This results in a potential drop from M1. The top plate of $C_{s}$ can only reach 1.6 V.
To prevent such a voltage drop, M1's gate voltage needs to have $V_{th,M1}$ higher than BL.
This would minimally need to be 2.4 V or higher, this process is commonly called bootstrapping.
\end{document}