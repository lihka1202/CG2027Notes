\documentclass{article}

\title{Assigment 2}
\author{Muthya Narayanachary Akhil}
\date{January 29, 2024}

\begin{document}
\maketitle

\section*{Question 1}
\subsection*{Part a}
As stated in the question, we are required to calculate $C_{wire}$, which can be expressed by the following equation:

\begin{equation}
    C_{wire} = C_{pp} + C_{fringe}
\end{equation}

\subsubsection*{Calculating parallel plate capacitance}
Based on the lecture notes, this can be calculated in the following way:
\begin{equation}
    C_{pp} = \frac{38 * 10^{-18}}{1 * 10^{-12}} * 250 * 10^{-6} * 0.4 * 10^{-6} = 3800 * 10^{-18}F = 3800aF
\end{equation}

\subsubsection*{Calculating fringing capacitance}
This can be found in the following way:
\begin{equation}
    C_{fringe} = \frac{13 * 10^{-18}}{10^{-6}} * 250 * 10^{-6} = 3250 * 10^{-18} = 3250aF
\end{equation}

Since fringe capacitance needs to include for both the side walls, the value needs to be doubled, that is:

\begin{equation}
    C_{fringe} = 3250 * 2 = 6500aF
\end{equation}

\subsubsection*{Calculating the capacitance of the wire}
Based on (2) and (4), we can determine the capacitance of the wire in the following way:
\begin{equation}
    C_{wire} = C_{pp} + C_{fringe} = 3800 + 6500 = 10300aF = 10.3fF
\end{equation}

\subsection*{Part b}
The goal is to calculate the $R_{wire}$, given that the sheet metal resistance ($R_{0}$) is 0.08$\frac{\Omega}{sq}$.
The width is given to to be 0.4$\mu$m, hence the goal is to fit as many squares of this size into the available length.
This can be done in the following way:

\begin{equation}
    R_{wire} = 0.08 * \frac{250 * 10^{-6}}{0.4 * 10^{-6}} = 50\Omega
\end{equation}

\section*{Question 2}
\subsection*{Part a}
We are given that the current flowing through the wire is 4mA. Hence the IR drop can be given by:
\begin{equation}
    IR_{drop} = 4 * 10^{-3} * 50 = 0.2 V
\end{equation}

\subsection*{Part b}
We are given that the voltage source has an output impedance of 50$\Omega$, and requried to calculate the propagation delay ($t_{p}$).
Based off the lecture notes, we are aware the the $t_{p}$ is a function of resistance when on $R_{on}$ and the load capacitance $C_{L}$.

\begin{equation}
    t_{p} = f(R_{on} \cdot C_{L}) = 0.69 * R_{on} * C_{L}
\end{equation}
Since there is an impedance from the voltage source, $R_{on}$ can be calculated to be:
\begin{equation}
    R_{on} = R_{wire} + Z_{source} = 50 + 50 = 100\Omega
\end{equation}

Based on (5) and (9) we can compute the $t_{p}$ to be the following:
\begin{equation}
    t_{p} = 100 * 10.3 * 10^{-15} = 7.107 * 10^{-13} = 71.07ps
\end{equation}

\section*{Question 3}

\end{document}