\documentclass{article}

\title{Assigment 1}
\author{Muthya Narayanachary Akhil}
\date{January 21, 2024}

\begin{document}
\maketitle
\section*{Question 1}
Based on the diagram, we can solve the sub parts in the following way:

\subsection*{Part a}

Rise time ($t_{r}$) is defined as the time taken for the voltage to rise from 0.1($V_{dd}$) to 0.9($V_{dd}$).
Fall time ($t_{f}$) is defined as the time taken for the voltage to to fall from 0.9($V_{dd}$) to 0.1($V_{dd}$).
\\\\Hence the rise time and fall time for the waveform $V_{out}$ can be calculated as:

\begin{equation}
    t_{r} = 383.86ps - 359.95ps = 23.91ps
\end{equation}

\begin{equation}
    t_{f} = 132.36ps - 113.35ps = 19.01ps
\end{equation}

\subsection*{Part b}
The high-to-low propagation delay ($t_{pHL}$) is defined as the time delay between 0.5($V_{in}$) in the rising edge and 0.5($V_{out}$) in the falling edge.
Based on the graph provided in the question, the following can be obtained:

\begin{equation}
    t_{pHL} = 120.88ps - 108.62ps = 12.26ps
\end{equation}

\subsection*{Part c}
The low-to-high propagation delay ($t_{pLH}$) is defined as the time delay between 0.5($V_{in}$) in the falling edge and 0.5($V_{out}$) in the rising edge.
Based on the graph provided in the question, the following can be obtained:

\begin{equation}
    t_{pLH} = 368.35ps - 355.05ps = 13.3ps
\end{equation}

\subsection*{Part d}
Based on the diagram, we can assume that the clock cycle in pico-seconds can be assumed to be $T_{in}$.
\\\\Hence:

\begin{equation}
    \frac{10}{100} * T_{in} = 355.05ps - 108.62ps = 246.43 ps
\end{equation}

Solving this for $T_{in}$ would result in the following:

\begin{equation}
    T_{in} = 246.43 ps * 10 = 2464.3ps
\end{equation}

The frequency can be calculated to be the following:

\begin{equation}
    f_{in} = \frac{1}{2464.3 * 10^{-12}} = 406 * 10^{6} Hz = 406MHz.
\end{equation}


\section*{Question 2}
Based on the notes, we note that $R_{on}$ is inversely proportional to $\frac{W}{L}$. This implies the following:

\begin{equation}
    R_{on} \propto \frac{1}{W}
\end{equation}

and

\begin{equation}
    R_{on} \propto L
\end{equation}

\subsection*{Part a}
We are given that $R_{on}$ is initially 100k$\Omega$. Based on the relationship mentioned in (8), we can determine the current value of $R_{on}$
\begin{equation}
    R_{on} = \frac{100}{5}k\Omega = 20k\Omega
\end{equation}

\subsection*{Part b}
Similarly, based on (9):

\begin{equation}
    R_{on} = 100k\Omega * 5 = 500k\Omega
\end{equation}


\end{document}